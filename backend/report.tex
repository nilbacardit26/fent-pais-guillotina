% LaTeX Template for short student reports.
% Citations should be in bibtex format and go in references.bib
\documentclass[a4paper, 11pt]{article}
\usepackage[top=3cm, bottom=3cm, left = 2cm, right = 2cm]{geometry} 
\geometry{a4paper} 
\usepackage[utf8]{inputenc}
\usepackage{textcomp}
\usepackage{graphicx} 
\usepackage{amsmath,amssymb}  
\usepackage{bm}  
\usepackage[pdftex,bookmarks,colorlinks,breaklinks]{hyperref}  
%\hypersetup{linkcolor=black,citecolor=black,filecolor=black,urlcolor=black} % black links, for printed output
\usepackage{memhfixc} 
\usepackage{pdfsync}
\usepackage{fancyhdr}
\usepackage{listings}

\pagestyle{fancy}

\title{Backend Fent País}
\author{Iskra Desenvolupament}
%\date{}

\usepackage{xcolor}

\definecolor{codegreen}{rgb}{0,0.6,0}
\definecolor{codegray}{rgb}{0.5,0.5,0.5}
\definecolor{codepurple}{rgb}{0.58,0,0.82}
\definecolor{backcolour}{rgb}{0.95,0.95,0.92}

\lstdefinestyle{python}{
    backgroundcolor=\color{backcolour},   
    commentstyle=\color{codegreen},
    keywordstyle=\color{magenta},
    numberstyle=\tiny\color{codegray},
    stringstyle=\color{codepurple},
    basicstyle=\ttfamily\footnotesize,
    breakatwhitespace=false,         
    breaklines=true,                 
    captionpos=b,                    
    keepspaces=true,                 
    numbers=left,                    
    numbersep=5pt,                  
    showspaces=false,                
    showstringspaces=false,
    showtabs=false,                  
    tabsize=2
}

\lstset{style=python}

\begin{document}
\maketitle
\tableofcontents

\section{Introducció}
Ens han encarregat renovar el backend de Fent País, i els nostre
arquitecte ens ha recomanat fer servir Guillotina. Fent País és un
servei que permet fer reserves de capses regals. Una capsa regal està
formada per N experiències, l'usuari pot triar una d'aquestes
experiències.

\section{Entorn de treball}
Farem servir un entorn virtual de python, necessitem python3, des del
directori on hi ha \textbf{setup.py}.

\begin{lstlisting}[language=Python, caption=Entorn virtual]
  cd backend
  pytho3 -m venv venv
  source venv/bin/activate
  pip install -r requirements.txt
  pip install -r requirements-test.txt
  pip install -e .
\end{lstlisting}
Si teniu un error a l'hora d'instal·lar els requeriments, probablement
haureu d'actuallitzar pip
\begin{lstlisting}[language=Python, caption=Entorn virtual]
  pip install --upgrade pip
\end{lstlisting}


Això instal·larà les dependències necessàries i el paquet de fentpais
en mode desenvolupament. El codi està tot fet, només haureu d'anar
descomentant les línies.

\section{Breu Explicació de traversal i guillotina}
Guillotina és un traversal. És com un arbre o similar al sistema de
fitxers UNIX, organitzats en carpetes i paths. En el cas de
guillotina, cada url mapeja un objecte diferent.

Guillotina té l'estructua bàsica següent:
\begin{lstlisting}
  db/container/{objecte}
\end{lstlisting}

\textbf{db}: La base de dades

\textbf{container}: La primera gran carpeta del sistema de tipus
IContainer. Sempre n'hi ha d'haver un com a mínim
\subsection{Arquitectura}
L'arquitectura més basica que fa servir guillotina és:
\begin{enumerate}
\item \textbf{Postgresql}: Base de dades relacional
\item \textbf{Redis}: Base de dades key value en memòria
\end{enumerate}
En el nostre cas, la postgres ens servirà de base de dades i de
catàleg, que vol dir que podrem indexar els nostres objectes per
cercar-los ràpidament. Podríem tenir un ElasticSearch també, però en
aquest cas farem servir la mateixa postgres.


\section{Requeriments del sistema}
\subsection{Traversal}
\begin{enumerate}
\item Creació de la carpeta \textbf{db/container/tipus\_capsa} quan es crea el container
\item Creació de la carpeta \textbf{db/container/reserves} quan es crea el container
\item Creació de l'endpoint {@}getExperiencies en el context de la
  carpeta tipus\_capsa \\ \textbf{db/container/tipus\_capsa/{@}getExperiencies}. Aquest
  endpoint ha de retornar totes les experiències de tots els tipus
  de capsa
\item Creació de l'endpoint {@}getExperiencies en el context de la
  carpeta foo\_tipus\_capsa \\
  \textbf{db/container/tipus\_cpasa/foo\_tipus\_capsa/{@}getExperiencies}. Aquest
  endpoint ha de retornar totes les experiències només de la capsa
  foo\_tipus\_capsa
\item Quan es crea una reserva, comprovar que el tipus de capsa
  existeix, si no existeix anul·lar la reserva i retornar un \textbf{412}
\item Un usuari/client només podrà crear reserves. No podrà afegir
  tipus\_capsa ni experiències, però sí que les podrà veure
\item Només un usuari root podrà crear tipus de capsa i experiencies
\item El sistema ha de tenir tests i testejar totes les funcionalitats
\end{enumerate}

\pagebreak
\section{Disseny}
\subsection{Traversal} 
Tenint en compte els requeriments del sistema, tindrem una estructura
de dades i de traversal com la següent.
\begin{lstlisting}[language=Python, caption=Estructura dades]
  /db/container/tipus_capsa/foo_tipus_capsa/foo_experiencia_1
  /db/container/tipus_capsa/foo_tipus_capsa/foo_experiencia_N
  /db/container/tipus_capsa/foo_tipus_capsa/@getExperiencies
  /db/container/tipus_capsa/@getExperiencies
  /db/container/tipus_capsa/foo_tipus_capsa_N/foo_experiencia_N
  /db/container/tipus_capsa/reserves/foo_reserva_1
  /db/container/tipus_capsa/reserves/foo_reserva_N
  /db/container/tipus_capsa/users/foo_user
\end{lstlisting}

Reserves i tipus\_capsa són carpetes, i dintre d'aquestes carpetes hi
haurà objectes de tipus reserva i experiència respectivament. La
carpeta users contindrà objectes de tipus User.

\subsection{Interfícies}
Les interfícies defineixen els camps dels objectes. També es poden
definir mètodes o corutines, sense la seva implementació. Les
interfícies ja estan definides dintre de la carpeta
\textbf{interfaces}

\subsubsection{Reserves: interfaces/reserves.py}
Definim les interfícies per la carpeta reserves i el tipus
reserva. Veim que la reserva només té un camp, el tipus\_capsa, que
serà l'id d'un objecte tipus\_capsa que referenciarà la capsa.
\begin{lstlisting}[language=Python, caption=Estructura dades]
  from guillotina import interfaces, schema

  class IReservesFolder(interfaces.IFolder):
    pass


  class IReserva(interfaces.IItem):
    tipus_capsa = schema.TextLine()
  \end{lstlisting}


\subsubsection{Experiencies: interfaces/experiencia.py}
\begin{lstlisting}[language=Python, caption=Estructura dades]
  class IExperiencia(interfaces.IItem):
    index_field("nom", type="text")
    nom = schema.TextLine(required=True)

    index_field("description", type="text")
    descripcio = schema.TextLine(required=True)

    categories = schema.List(
        title="Categories",
        description="Categories de la experiencia",
        value_type=schema.TextLine(),
        required=False
    )
  \end{lstlisting}
  Experiència tindrà un nom, una descripció i una llista de
  categories. El mètode index\_field indexa aquest camp al catàleg, en
  aquest cas a la Postgres


  El següent codi indexa el camp categories, que és una llista, com un
  string de paraules. Normalment, sempre que indexem dades més
  complexes, les serialitzem
  \begin{lstlisting}[language=Python, caption=Estructura dades]
  @directives.index_field.with_accessor(
    IExperiencia,
    "categories",
    type="text",
    field="categories",
    store=True
    )
    async def index_tests_respondre(obj):
       results = ""
    for category in obj.categories:
       results = results + category
       if results != "":
           return results
         \end{lstlisting}
         @directives no és una interfície com a tal, podríem
         definir-la en un altre fitxer, però en aquest cas ho he fet
         en el mateix fitxer \textbf{experiencia.py}

\subsubsection{Experiencies: interfaces/tipus\_capsa.py}
    \begin{lstlisting}[language=Python, caption=Estructura dades]
  from guillotina import interfaces, schema


class ITipusCapsaFolder(interfaces.IFolder):
    async def get_experiencies():
        """
        Get all experiences
        """


class ITipusCapsa(interfaces.IFolder):
    async def get_experiencies():
        """
        Get experiences of the capsa
        """

    nom = schema.Text(required=True)

    descripcio = schema.TextLine(required=True)

    preu = schema.Float(required=True)
  \end{lstlisting}

  Tipus capsa té un nom, una descripció i un preu. També definim la
  interfície de dos corutines, get\_experiencies pels dos objectes.


\pagebreak

\section{Desenvolupament}
En aquesta secció anirem descomentant el codi, i provant les diferents
funcionalitats.
\subsection{Definició del container}
Primer definirem quins tipus hi poden haver dintre del container. Com
hem vist en els requeriments, les carpetes que podrem afegir són
ReservesFolder i TipusCapsaFolder. Per fer-ho descomentarem les línies
que hi ha entre el comentari \textbf{TODO\_1} i \textbf{END\_TODO\_1}
del fitxer \textbf{container.py}, fent servir la directiva
\textbf{{@}configure.contenttype} per configurar el tipus, el camp
\textbf{allowed\_types}. El camp behaviors ens permet tenir camps
extres que ja venen configurats pel mateix behavior.

\begin{lstlisting}[language=Python, caption=Definició container]
@configure.contenttype(
    type_name="Container",
    schema=IContainer,
    behaviors=[
        'guillotina.behaviors.dublincore.IDublinCore'
    ],
    allowed_types=["ReservesFolder", "TipusCapsaFolder"]
)
class Container(Container):
  pass
\end{lstlisting}


\subsection{Creació de les carpetes reserves i tipus\_capsa}
Començarem creant les dues carpetes reserves i tipus\_capsa
automàticament quan es crea el container. Per fer això descomentarem
les línies que hi ha entre el comentari \textbf{TODO\_2} i
\textbf{END\_TODO\_2} del fitxer \textbf{container.py}

Si us hi fixeu, hi ha un decorador configurant un subscriber:
\begin{lstlisting}[language=Python, caption=Subscriber container]
  @configure.subscriber(for_=(IContainer, IObjectAddedEvent))
\end{lstlisting}
Això fa executar la corutina \textbf{creacio\_container} sempre que un objecte
de tipus IContainer es crea.
  
Les línies que heu de descomentar el que fan és crear la carpeta amb
id reserves del tipus ReservesFolder. Finalment notifiquem que hem
creat una carpeta de tipus ReservesFolder, perquè altres subscribers
es puguin beneficiar de la creació de la carpeta.
\begin{lstlisting}[language=Python, caption=Creació reserves folder]
  reserves_folder = await create_content_in_container(
        parent=obj,
        type_="ReservesFolder",
        check_security=False,
        id_="reserves",
        title="Reserves' folder"
    )
    await notify(ObjectAddedEvent(reserves_folder, reserves_folder, reserves_folder.id))
\end{lstlisting}

El següent codi crea la carpeta tipus\_capsa del tipus TipusCapsaFolder
\begin{lstlisting}[language=Python, caption=Creació tipus\_capsa]
  tipus_capsa_folder = await create_content_in_container(
        parent=obj,
        type_="TipusCapsaFolder",
        check_security=False,
        id_="tipus_capsa",
        title="Tipus capses' folder"
    )
    await notify(ObjectAddedEvent(tipus_capsa_folder, tipus_capsa_folder, tipus_capsa_folder.id))
  \end{lstlisting}

  Ara en el fitxer \textbf{content/reserves.py}, hem de definir el content de tipus
  ReservesFolder. Per fer això descomentem el que hi ha entre
  \textbf{TODO\_3} i \textbf{END\_TODO\_3}

  \begin{lstlisting}[language=Python, caption=Configuració content ReservesFolder]
  @configure.contenttype(
    type_name="ReservesFolder",
    schema=IReservesFolder,
    behaviors=[
        'guillotina.behaviors.dublincore.IDublinCore'
    ],
    allowed_types=["Reserva"],
    globally_addable=False
)
class ReservesFolder(content.Folder):
    async def get_reserves(self):
        search_instance = query_utility(ICatalogUtility)
        return search_instance.search({"type_name": "Reserva"})
      \end{lstlisting}

  Ara en el fitxer \textbf{content/tipus\_capsa.py}, hem de definir el content de tipus
  TipusCapsaFolder. Per fer això descomentem el que hi ha entre
  \textbf{TODO\_4} i \textbf{END\_TODO\_4}

  \begin{lstlisting}[language=Python, caption=Configuració content ReservesFolder]
    @configure.contenttype(
    type_name="TipusCapsaFolder",
    schema=ITipusCapsaFolder,
    behaviors=[
        'guillotina.behaviors.dublincore.IDublinCore'
    ],
    allowed_types=["TipusCapsa"],
    globally_addable=False
    )
    class TipusCapsaFolder(content.Folder):
    async def get_experiencies(self):
        search_instance = query_utility(ICatalogUtility)
        return await search_instance.search(
            context=self,
            query={"type_name": "Experiencia"}
)
\end{lstlisting}
Aquest codi defineix el tipus TipusCapsaFolder, i també implementa la
corutina get\_experiences, que fent servir el cataleg, fa una cerca per
retornar totes les experiències de totes les capses
  

  Un cop descomentades les línies, podem passar el test que comprova
  que ho hem fet correctament, en el directori de la carpeta
  backend/fentpais, correm pytest:
  \begin{lstlisting}[language=Python, caption=Testing 1]
    DATABASE=pg pytest tests/test_container.py
  \end{lstlisting}
  
  Aquest test crea un container i comprova que les carpetes s'han
  creat correctament. Si passa el test, felicitats, les carpetes
  reserves i tipus\_capsa han estat creades en la creació del
  container. Si no és així, torneu a mirar les línies que estan dintre
  dels blocs TODO\_1, TODO\_2, TODO\_3 i TODO\_4 container.py,
  reserves.py i tipus\_capsa.py respectivament.

\subsection{Creació d'una capsa dintre de la carpeta tipus\_capsa}
En el fitxer \textbf{content/tipus\_capsa.py}, hem de definir el content de tipus
  TipusCapsa. Per fer això descomentem el que hi ha entre
  \textbf{TODO\_5} i \textbf{END\_TODO\_5}

\begin{lstlisting}[language=Python, caption=Definició content type TipusCapsa]
@configure.contenttype(
    type_name="TipusCapsa",
    schema=ITipusCapsa,
    allowed_types=["Experiencia"],
    behaviors=[
        'guillotina.behaviors.dublincore.IDublinCore'
    ],
    globally_addable=False
)
class TipusCapsa(content.Folder):
    async def get_experiencies(self):
        search_instance = query_utility(ICatalogUtility)
        return await search_instance.search(
            context=self,
            query={"type_name": "Experiencia", "metadata": "*"}
)
\end{lstlisting}

Aquest codi defineix el tipus TipusCapsa, i també implementa la
corutina get\_experiences, que fent servir el cataleg, fa una cerca per
retornar les experiències que hi ha dintre de la capsa

Un cop descomentades les línies, podem passar el test que comprova
  que ho hem fet correctament, en el directori de la carpeta
  backend/fentpais, correm pytest:
  \begin{lstlisting}[language=Python, caption=Testing 2]
    DATABASE=pg pytest tests/test_capsa.py -k test_creacio_capsa
  \end{lstlisting}

\subsection{Creació d'una experiència dintre d'una capsa}
En el fitxer \textbf{content/experiencia.py}, hem de definir el content de tipus
  TipusCapsa. Per fer això descomentem el que hi ha entre
  \textbf{TODO\_6} i \textbf{END\_TODO\_6}

\begin{lstlisting}[language=Python, caption=Definició content type Experiencia]
@configure.contenttype(
    type_name="Experiencia",
    schema=IExperiencia,
    behaviors=[
        'guillotina.behaviors.dublincore.IDublinCore'
    ],
    globally_addable=False
)
class Experiencia(content.Item):
    pass
\end{lstlisting}

Un cop descomentades les línies, podem passar el test que comprova
  que ho hem fet correctament, en el directori de la carpeta
  backend/fentpais, correm pytest:
  \begin{lstlisting}[language=Python, caption=Testing 3]
    DATABASE=pg pytest tests/test_capsa.py -k test_creacio_experiencia
  \end{lstlisting}

\subsection{Creació dels endpoints @getExperiencies}
En el fitxer \textbf{content/tipus\_capsa.py}, hem de definir els
endpoints \textbf{{@}getExperiencies} descomentant les línies de codi
entre els comentaris \textbf{TODO\_7} i \textbf{END\_TODO\_7}

\begin{lstlisting}[language=Python, caption=Definició endpoints]
@configure.service(context=ITipusCapsa, name="@getExperiences", method="GET", permission="guillotina.ViewContent")
@configure.service(context=ITipusCapsaFolder, name="@getExperiences", method="GET", permission="guillotina.ViewContent")
class GetExperiencesCapsa(Service):
    async def __call__(self):
        return await self.context.get_experiencies()
\end{lstlisting}
      
El \textbf{{@}configure.service} defineix els endpoints per dos contexts
diferents, un per la carpeta tipus\_capsa i capsa. Fixeu-vos en un
detall, self.context pot ser un objecte de TipusCapsaFolder o
TipusCapsa, com que hem implementat les dues corutines amb el mateix
nom, podem aprofitar la mateixa crida get\_experiencies. Mireu també
el permission: \textbf{guillotina.ViewContent}, només podrà cridar aquest
endpoint qui tingui aquest permís.
      
      
Un cop descomentades les línies, podem passar el test que comprova
  que ho hem fet correctament, en el directori de la carpeta
  backend/fentpais, correm pytest:
\begin{lstlisting}[language=Python, caption=Testing 4]
  DATABASE=pg pytest tests/test_capsa.py -k test_tipus_capsa_get_experiencies_endpoint
\end{lstlisting}

\subsection{Creació de les reserves}
En el fitxer \textbf{content/reserves.py}, hem de definir el content Reserva, descomentant el codi entre
\textbf{TODO\_8} i \textbf{END\_TODO\_8}

\begin{lstlisting}[language=Python, caption=Definició content type Reserva]
@configure.contenttype(
    type_name="Reserva",
    schema=IReserva,
    behaviors=[
        'guillotina.behaviors.dublincore.IDublinCore'
    ],
    globally_addable=False
)
class Reserva(content.Item):
    async def check_tipus_capsa(self):
        search_instance = query_utility(ICatalogUtility)
        container = get_current_container()
        tipus_capsa_folder = await container.async_get("tipus_capsa")
        results = await search_instance.unrestrictedSearch(context=tipus_capsa_folder, query={"id": self.tipus_capsa})
        if results["items_total"] == 0:
            raise HTTPPreconditionFailed()
\end{lstlisting}
En el codi configurem el contingut Reserva, i creem un mètode que
comprova que el camp tipus\_capsa, que és un id que referencia a una
capsa, existeix. Si no existeix retornem un 412. Guillotina és
transaccional, si aixequem una excepció, la transacció s'anul·la, i
l'objecte no es crearà

Si ho hem fet bé podrem passar el test:
\begin{lstlisting}[language=Python, caption=Testing 5]
  DATABASE=pg pytest tests/test_reserva.py -k test_creacio_experiencia_correcte
\end{lstlisting}

En el fitxer \textbf{content/reserves.py}, ara definirem el subscriber
que comprova que quan es crea o es modifica una reserva, el
tipus\_capsa és corecte. Descomentem el codi entre \textbf{TODO\_9} i
\textbf{END\_TODO\_9}

\begin{lstlisting}[language=Python, caption=Definició subscriber Reserva]
@configure.subscriber(for_=(IReserva, IObjectAddedEvent))
@configure.subscriber(for_=(IReserva, IObjectModifiedEvent))
async def creacio_modificacio_reserva(obj, event):
    await obj.check_tipus_capsa()
  \end{lstlisting}

Si ho hem fet bé podrem passar el test:
\begin{lstlisting}[language=Python, caption=Testing 6]
  DATABASE=pg pytest tests/test_reserva.py -k test_creacio_experiencia_412
\end{lstlisting}

          

\pagebreak

\section{Evaluation}

We did some experiments \ldots

\pagebreak

\section{Conclusions and Future Work}

From our experiments we can conclude that \ldots

\bibliographystyle{abbrv}
% \bibliography{references}  % need to put bibtex references in references.bib 
\end{document}